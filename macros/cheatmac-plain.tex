% set up PDF output here, so we won't pollute the source
\pdfoutput=1
\pdfpkresolution=600

\input ucwmac
\input ucw-ofs

% \input color alone works on modern software, but this works on older machines also
\input miniltx
\input color.sty

\language\czech

\landscape

\newcount\columnsperpage

\columnsperpage=3

\def\versionnumber{0.01}  % Version of this reference card
\def\year{2012}
\def\month{January}
\def\version{\month\ \year\ v\versionnumber}

% make \bye not \outer so that the \def\bye in the \else clause below
% can be scanned without complaint.
\def\bye{\par\vfill\supereject\end}

\newdimen\intercolumnskip
\newbox\columna
\newbox\columnb

\def\ncolumns{\the\columnsperpage}

\message{[\ncolumns\space
   column\if 1\ncolumns\else s\fi\space per page]}

\def\scaledmag#1{ scaled \magstep #1}

% This multi-way format was designed by Stephen Gildea
% October 1986.
% Modified for Czech fonts.

\if 1\ncolumns
   \hsize 4in
   \vsize 10in
   \voffset -.7in
   \font\titlefont=\fontname\tenbf \scaledmag3
   \font\headingfont=\fontname\tenbf \scaledmag2
   \font\smallfont=\fontname\sevenrm
   \font\smallsy=\fontname\sevensy

   \footline{\hss\folio}
   \def\makefootline{\baselineskip10pt\hsize6.5in\line{\the\footline}}
\else
   \hsize 3.56in
   \vsize 7.75in
   \hoffset -.75in
   \voffset -.745in
   \font\titlefont=csbx10 \scaledmag2
   \font\headingfont=csbx10 \scaledmag1
   \font\smallfont=csr6
   \font\smallsy=cssy6
   \font\eightrm=csr8
   \font\eighti=csmi8
   \font\eightsy=cssy8
   \font\eightbf=csbx8
   \font\eighttt=cstt8
   \font\eightit=csti8
   \font\eightsl=cssl8
   
   \textfont0=\eightrm
   \scriptfont0=\sevenrm
   \scriptscriptfont0=\fiverm
   \textfont1=\eighti
   \textfont2=\eightsy
   \def\rm{\fam0 \eightrm}
   \def\bf{\eightbf}
   \def\tt{\eighttt}
   \def\it{\eightit}
   \def\sl{\eightsl}
   \font\msbm=msbm8
   \def\mathbb{\msbm}
   
   \normalbaselineskip=.8\normalbaselineskip
   \normallineskip=.8\normallineskip
   \normallineskiplimit=.8\normallineskiplimit
   \normalbaselines\rm          %make definitions take effect

   \if 2\ncolumns
     \let\maxcolumn=b
     \footline{\hss\rm\folio\hss}
     \def\makefootline{\vskip 2in \hsize=6.86in\line{\the\footline}}
   \else \if 3\ncolumns
     \let\maxcolumn=c
     \nopagenumbers
   \else
     \errhelp{You must set \columnsperpage equal to 1, 2, or 3.}
     \errmessage{Illegal number of columns per page}
   \fi\fi

   % \intercolumnskip=.46in
   \intercolumnskip=.20in
   \def\abc{a}
   \output={%
       % This next line is useful when designing the layout.
       %\immediate\write16{Column \folio\abc\space starts with \firstmark}
       \if \maxcolumn\abc \multicolumnformat \global\def\abc{a}
       \else\if a\abc
        \global\setbox\columna\columnbox \global\def\abc{b}
         %% in case we never use \columnb (two-column mode)
         \global\setbox\columnb\hbox to -\intercolumnskip{}
       \else
        \global\setbox\columnb\columnbox \global\def\abc{c}\fi\fi}
   \def\multicolumnformat{\shipout\vbox{\makeheadline
       \hbox{\box\columna\hskip\intercolumnskip
         \box\columnb\hskip\intercolumnskip\columnbox}
       \makefootline}\advancepageno}
   \def\columnbox{\leftline{\pagebody}}

   \def\bye{\par\vfill\supereject
     \if a\abc \else\null\vfill\eject\fi
     \if a\abc \else\null\vfill\eject\fi
     \end}
\fi

\parindent 0pt
\parskip 1ex plus .5ex minus .5ex

\def\small{\smallfont\textfont2=\smallsy\baselineskip=.8\baselineskip}

\outer\def\newcolumn{\vfill\eject}

\outer\def\title#1{{\titlefont\centerline{#1}}\vskip 1ex plus .5ex minus.5ex}

\outer\def\section#1{\par\goodbreak
   \vskip .75ex plus 1ex minus 2ex {\headingfont #1}\mark{#1}%
   \vskip .5ex plus .5ex minus .5ex}
\outer\def\subsection#1{\par
   \vskip .75ex plus 1ex minus 2ex {\bf #1}\mark{#1}%
   \vskip .5ex plus .5ex minus .5ex}
\outer\def\subsubsection#1{\par
{\bf #1}\par}

% fake section -- looks the same, but does not force itself
% to a new column that much
\outer\def\fsection#1{\par
   \vskip .75ex plus 1ex minus 2ex {\headingfont #1}%
   \vskip .5ex plus .5ex minus .5ex}
\def\paralign{\vskip\parskip\halign}

%\def\<#1>{$\langle${\rm #1}$\rangle$}

\def\begintext{\par\leavevmode\begingroup\parskip0pt\rm}
\def\endtext{\endgroup}

% smaller indenting of itemize lists for a compact format
\itemindent=0.15in
\itemnarrow=0in

% math-related definitions (compressed than usual):
\def\slightnegindent{\parshape 2 -3pt \hsize 0in \hsize}

% FONTS: script font
\font\tenscr=rsfs10   % The names \tenscr, \sevenscr, \fivescr can
\font\sevenscr=rsfs7  % be any command names, but should be unique and
\font\fivescr=rsfs5   % descriptive.

% Allocate the math family...
\newfam\scrfam        % The name \scrfam can be any command, but should
                      % be unique and descriptive.
% ...and assign the fonts:
\textfont\scrfam=\tenscr
\scriptfont\scrfam=\sevenscr
\scriptscriptfont\scrfam=\fivescr

% I don't know if this is correct or necessary:
\skewchar\tenscr=127
\skewchar\sevenscr=127
\skewchar\fivescr=127

% Define a command to switch to this new math family:
\def\scr{\fam\scrfam}

% FONTS: blackboard bold


% We pass colour and statement of the particular statement
% through global macros, so we can keep variable arguments.

% D: and similar
\long\def\statementshort#1{%
{\bf\begingroup\curcolor\curname: \endgroup}%
#1}

% D(theorem):
\long\def\statementstandard[#1]#2{%
{\begingroup\bf\curcolor\curname(\endgroup{#1}\begingroup\curcolor): \endgroup}%
#2}

% |D(theorem):
% \long\def\statementlong#1#2{{\bf \begingroup\csname#1color\endcsname
% \csname#1name\endcsname(\endgroup{\i
% #2}\begingroup\csname#1color\endcsname):\endgroup}}




%\def\statement#1#2{\slightnegindent \claimitem{#1}{#2}}

\long\def\statement{%
    \futurelet\statementToken\statementDecide
}

\long\def\statementDecide{%
    \ifx\statementToken [%
        \let\next = \statementstandard
    \else
        \let\next = \statementshort
    \fi
    \next
}


\def\dfn{%
    \def\curname{D}
    \def\curcolor{\color[rgb]{0.8,0,0}}
    \statement
}

\def\thm{%
    \def\curname{T}
    \def\curcolor{\color[rgb]{0,0,0.5}}
    \statement
}

\def\lem{%
    \def\curname{L}
    \def\curcolor{\color[rgb]{1,0,1}}
    \statement
}

\def\prf{%
    \def\curname{P}
    \def\curcolor{\color[rgb]{0,0.5,0}}
    \statement
}

\def\obs{%
    \def\curname{O}
    \def\curcolor{\color{black}}
    \statement
}

\def\res{%
    \def\curname{R}
    \def\curcolor{\color{black}}
    \statement
}

\def\cor{%
    \def\curname{C}
    \def\curcolor{\color{black}}
    \statement
}

\def\opn{%
    \def\curname{Q}
    \def\curcolor{\color{black}}
    \statement
}

\def\nta{%
    \def\curname{N}
    \def\curcolor{\color{black}}
    \statement
}

\def\clm{%
    \def\curname{Clm}
    \def\curcolor{\color{black}}
    \statement
}

\def\andamp{\mathop\&}

\def\st{{\rm\ s.t.\ }}
\def\iff{\leftrightarrow}
\def\then{\rightarrow}
\def\neht{\leftarrow}
\def\rng{{\rm Rng}}
\def\dom{{\rm Dom}}
\def\var{{\rm Var}}
\def\mod{{\ \rm mod\ }}

\def\Natu{{\mathbb N}}

\def\TODO{{\bf TODO:}}
\def\term#1{{\it #1}}
% my compact notation for $x_1, x_2, \dots, x_n$.
\def\mls#1#2{#1_{[1,#2]}}

% enumerability-related macros
\def\da{\downarrow}
\def\ua{\uparrow}
\def\daeq{\downarrow=}
\def\daneq{\downarrow\neq}

% handout-related macros -- author, presenter
\def\author#1{\centerline{\bf #1}}
\def\presenter#1{\centerline{\it Presented by #1}}

