\input cheatmac

\font\tenscr=rsfs10   % The names \tenscr, \sevenscr, \fivescr can
\font\sevenscr=rsfs7  % be any command names, but should be unique and
\font\fivescr=rsfs5   % descriptive.

% Allocate the math family...
\newfam\scrfam        % The name \scrfam can be any command, but should
                      % be unique and descriptive.
% ...and assign the fonts:
\textfont\scrfam=\tenscr
\scriptfont\scrfam=\sevenscr
\scriptscriptfont\scrfam=\fivescr

% I don't know if this is correct or necessary:
\skewchar\tenscr=127
\skewchar\sevenscr=127
\skewchar\fivescr=127

% Define a command to switch to this new math family:
\def\scr{\fam\scrfam}
%%% end

\def\hyp#1{{\cal #1}}

\def\ind{{\cal I}}

\def\Con{{\rm Con}}
\def\state{{\rm state}}
\def\OPT{{\rm OPT}}
\def\path{{\rm path}}
\def\univ{{\rm univ}}
\def\pw{{\rm pw}}
\def\F{\hyp{F}}
\def\X{\hyp{X}}

\def\author#1{\centerline{\bf #1}}
\def\presenter#1{\centerline{\it Presented by #1}}

\title{Improved approximation for 3D}
\title{matching via bounded pathwidth}
\title{local search}
\author{Marek Cygan}
\presenter{Martin Böhm}
\centerline{{\it Approximation and Online Algorithm Seminar 2014, MFF UK}}

\section{Definitions and Notation}

\dfn[{\csc $k$-Set-Packing}]{

{\bf Input:} A family $\F ⊆ 2^U$ of sets of size at most $k$.

{\bf Goal:} Find a maximum size subfamily of $\F$ of pairwise disjoint sets.
}

\thm[Main Theorem]{For any $ε>0$ and any integer $k ≥ 3 $ there is a polynomial
time $(k+1+ε)/3$-approximation algorithm for {\csc $k$-Set-Packing}.}

\nta{ $\overline{\F_0} ≡ \F ∖ \F_0$.} 

\nta{ For a vertex set $\X$,  $N(\X) ≡ $ neighbors of vertices in $\X$.}

\nta{ For a vertex set $\X$, $N[\X] ≡ N(\X) ∪ \X$.}

\dfn[Pathwidth]{A graph $G$ has pathwidth at most $\pw$ if
it has a tree-decomposition of treewidth at most $\pw$ where
the decomposition itself is a path.}

\dfn[Conflict graph]{For a disjoint starting family $\F_0 ⊆ \F$
we define a {\it conflict graph} $\Con_{\F_0}$ as a bipartite
graph with vertex set $\F$ and edge set
$\{ S_1S_2 | S_1 ∈ \F_0, S_2 ∈ \overline{\F_0}, S_1 ∩ S_2 ≠ ∅\}$.

We will use $\Con$ if the starting family is clear from context. We lose some
information in the conflict graph -- namely the disjointness
information for neighbors in $\overline{\F_0}$.
}

\dfn[Improving set\dots]{For a starting family $\F_0$ we call an {\it improving set} $\X$ a set of vertices of $\overline{\F_0}$
such that:
\itemize\ibull
\: All members of $\X$ are pairwise disjoint;
\: $|N(\X)| < |\X|$, i.e. we can improve $\F_0$ using $\X$.
\endlist 
}

\dfn[\dots of bounded pathwidth]{ An improving set $\X$ with respect to $\hyp{F_0} ⊆ \F$ has {\it pathwidth at most $\pw$} if the subgraph
of the conflict graph induced by $N[\X]$ is of pathwidth at most $\pw$.
}

\section{Part 1 -- The FPT Algorithm}

{\bf Color coding:}
\itemize\ibull
\: A dynamic-programming technique used to efficiently find a small structure of bounded treewidth (a path, cycle, etc.) within a larger graph.
\: Idea: Have a coloring function assign colors to vertices. Look only for the substructure
that is {\it colorful}.
\: Originally probabilistic: if the probability that the structure becomes colorful is
non-trivial, we try many coloring functions and get a polynomial, constant-error algorithm.
\: Can be derandomized by a standard argument.
\endlist

\lem[Bounded pathwidth algorithm]{Let $\pw$, $k$ (parameter of {\csc $k$-Set-Packing}) and $r$ (size of the improving set we look for, later set to $O(\log |\F|)$) be fixed.
There exists an algorithm that:
\itemize\ibull
\: given a disjoint family $\F_0 ⊆ \F$ and two coloring functions $c_{\path} : \F_0 → [r-1]$ and $c_{\univ}: U → [rk]$,
\: in time $2^{O(rk)}|\F|^{O(\pw)}$,
\: determines whether an improving set $\X$ of size at most $r$ and pathwidth at most $\pw$ exists, s.t. $c_{\path}$ is injective
on $N_{\Con}(\X)$ and $c_{\univ}$ is injective on $⋃_{S ∈ \X} S$.
\endlist
}

\prf{Create an auxiliary digraph $G_{\state}$ of size
$O(2^{r(k+1)}|\F|^{\pw+1})$.
Every state (vertex of $G_{\state}$) will represent a partial pathwidth decomposition.
We will traverse this graph and look for a pathwidth decomposition that is also
an improving set.

Instead of the entire partial pathwidth decomposition, we store in every state
a triplet $(D_{\path}, D_{\univ}, B)$, where
\itemize\ibull
\: $D_{\path}$ are the colors of members of $\F_0$ we have already traversed in the decomposition; 
\: $D_{\univ}$ are the colors of the universe that we have seen so far (inside sets of $\overline{\F_0}$
that we have traversed);
\: $B$ is a set of size at most $\pw+1$ -- our current pathwidth decomposition bag.
\endlist

We add directed edges to $G_{\state}$ which correspond to progressing along a
pathwidth decomposition. We then run a graph search on $G_{\state}$.

The injectiveness of $D_{\path}$ ensures that we do not go back in the
pathwidth decomposition; the injectiveness of $D_{\univ}$ ensures that the visited sets of $\overline{\F_0}$ are disjoint.
}

\clm{There exists a path in the graph $G_{\state}$ from the vertex $(∅,∅,∅)$ to the vertex
$(D_{\path}, D_{\univ}, ∅)$ for $D_{\path} < D_{\univ}/k$ if and only if there
exists an improving set $\X$ of size at most $r$ of pathwidth at most $\pw$, such that
$D_{\path}$ is injective on $N(\X)$ and $D_{\univ}$ is injective on $⋃_{S∈\X} S$.
}

\section{Part 2 -- Constant Pathwidth Suffices}


\thm[Main claim]{Let $k$ be an integer, $ε>0$. Then there exist constants $c_1(k,ε), c_2(k,ε)$ such that for any disjoint
family $\F_0 ⊆ \F$ for which there is no improving set of size at most $c_1 \log n$ that has pathwidth at most
$c_2$, we have $|OPT| ≤ ((k+1)/3 + ε)|\F_0|$.
}

\prf{Assume we are in a situation where there is no valid improving set.
Set $C ≡ \OPT ∩ \F_0$, $A_0 ≡ \F_0 ∖ C$, $B_0 ≡ \OPT ∖ C$. We restrict ourselves
to $G[A_0 ∪ B_0]$. We will create
a sequence of $1/ε$ subgraphs $G[A_i ∪ B_i]$ which have roughly the same properties
as $(A_0, B_0)$, that is:

\itemize\ibull
\: in $G[A_i ∪ B_i]$ there is no subset $\X ⊆ B_i$ of size at most $2(k+1)^{1/ε -i}$ such that $|N(\X)| < |\X|$.
\: $|A_0 ∖ A_i| = |B_0 ∖ B_i|$, or equivalently $|A_0 ∖ B_0| = |A_i ∖ B_i|$.
\endlist

Split $B_i$ into sets $B_i^1$, $B_i^2$, $B_i^{≥3}$, where the superscript indicates
the degree of the vertices.

We note the following two claims:

\obs{Either $|B_i^1| ≤ ε|\OPT| ≤ ε |A_i|$ or we can construct $G[A_{i+1} ∪ B_{i+1}]$.}

\clm[Key claim]{$B_i^2$ always satisfies $|B_i^2| ≤ (1+ε)|A_i|$.}

If the two claims hold, we note that the number of edges satisfies
$||G[A_i ∪ B_i]|| ≥ 1|B_i^1| + 2|B_i^2| + 3|B_i^3|$, but it also satisfies
$||G[A_i ∪ B_i]|| ≤ k|A_i|$. Summing up all inequalities together, we get
$|B_i| ≤ ((k+1)/3 + ε)|A_i|$ and finally $|\OPT| ≤ ((k+1)/3 + ε)|\F_0|$.
}

\prf[Key claim]{Restrict the graph to only $G ≡ G[A_i ∪ B_i^2]$. For contradiction,
assume $|B_i^2| > (1+ε)|A_i|$.

The graph $G$ is a bipartite graph where every vertex of the partition
$B_i^2$ has degree exactly two. We can look at this graph as a
multigraph $G'$ which has the vertex set $A_i$ and edge set $B_i^2$
(understood as pairs of $A_i$).

$|B_i^2| > (1+ε)|A_i|$ implies ${||G'||\over |G'|} = 1+ε$, which implies
$d(G') = 2+2ε$. We find an improving set of size $O(\log G') = O(\log \F)$ and
of constant pathwidth using Overcharged short cycle lemma below.
}

\lem[Short cycle lemma]{Let $G$ be a graph of minimal degree $3$. Then $G$ contains
a cycle of length at most $O(\log n)$.
}

\lem[Overcharged short cycle lemma]{Let $H$ be a multigraph, $|H|=n$, $δ(H) ≥ 3$. Let $w_e$ be a labeling on edges of $H$ by a subset of
some alphabet $Σ$. Assume that for some $γ$, the following holds: $∀e ∈ E: w_e ≤ γ$ and $∀α ∈ Σ: |\{$occurences of $α$ in all labels$\}|≤γ$.

Then there exists a subtree $T_0 = (V_0,E_0)$ with root $r$ and two edges $e_1$ and $e_2$ outside $T_0$ such that:
\itemize\ibull
\: $|V_0| ≤ 4(\log_{3/2} n + 2)$,
\: both $T_0+e_1$ and $T_0+e_2$ contains a cycle,
\: $T_0$ is a tree with at most 4 leaves, 
\: If we set $β≡⌈\log_{3/2}(12γ^2)⌉$, then every pair of edges $e_a, e_b ∈ T_0$ it holds that if their labels intersect, then
$|dist(e_a,r) - dist(e_b,r)| ≤ β$, where for an edge $e=uv$, $dist(e,r) ≡ \min(dist(u,r),dist(v,r))$.
\endlist
}

\nta{Denote $P$ as the subtree of $T_0$ such that $P$ is connected and it contains all endpoints
of $e_1$ and $e_2$.}

\obs{$P+e_1+e_2$ has $n$ vertices, but $n+1$ edges.}

\cor{Let $T_0$, $e_1, e_2$ be as in Overcharged short cycle lemma. Then the graph $P + e_1 + e_2$ has a path decomposition
of width at most $4β + 3$.}

\bye
