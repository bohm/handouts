\input macros/cheatmac

\usepackage{bm} % bold math

% left align
\usepackage{environ}
\makeatletter
\NewEnviron{Lalign}{\tagsleft@true\begin{alignat*}{1}\BODY\end{alignat*}}
\makeatother

\usepackage{algorithm}
\usepackage[noend]{algpseudocode}
% \usepackage{microtype}
\usepackage{accents}
\usepackage{varwidth}
\usepackage{tikz}

% small spacing in align
% \abovedisplayskip=0pt plus 3pt
% \abovedisplayshortskip=0pt plus 3pt
% \belowdisplayskip=0pt plus 3pt 
% \belowdisplayshortskip=0pt plus 3pt

\newcommand{\netflow}[2]{\nabla #1_{#2}}

\newcommand{\Qtwo}{\mathbb{Q}_2}
\DeclareMathOperator{\bits}{bits}
\newcommand{\encl}[1]{\| #1\|}
\DeclareMathOperator{\binenc}{bin}
\newcommand{\encb}[1]{\binenc(#1)}
\newcommand{\Z}{\mathbb{Z}}
\newcommand{\Q}{\mathbb{Q}}
\newcommand{\R}{ℝ}
\newcommand{\Vm}{V \setminus \{t\}}
\newcommand{\VN}{V^-}
\newcommand{\VP}{V^+}
\newcommand{\VZ}{V^0}
\newcommand{\globpot}{\Xi}
\newcommand{\plenty}[1]{3n(d_{#1}+1)}

%\newcommand{\Sc}{.}
%\newcommand{\Ss}{.}
\newcommand{\Sc}{S}
\newcommand{\Ss}{\bar{S}}
\newcommand{\Bars}{\bar{S}}

\renewcommand{\algorithmicrequire}{\textbf{Input:}}
\renewcommand{\algorithmicensure}{\textbf{Output:}}

\newcommand{\instance}{\mathcal{I}}

\newcommand{\kosher}{anchored}
\newcommand{\solid}{solid}

%\newcommand{\anchors}{anchors}
%\newcommand{\anchor}{anchor}


%\newcommand{\ole}{\overleftarrow}
\newcommand{\ole}[1]{\accentset{\leftarrow}{#1}}
%\newcommand{\obe}[1]{\overleftrightarrow{#1}}
\newcommand{\obe}[1]{\accentset{\leftrightarrow}{#1}}



\def\Bigpi{{\rm Par}}
\def\intcone{{\rm intcone}}
\def\cone{{\rm cone}}
\def\conv{{\rm conv}}
\def\vert{{\rm vert}}
\def\Las{{\textsc{Las}}}
\def\Oh{{\rm O}}
\def\supp{{\rm supp}}
\def\enc{{\rm enc}}

\DeclareMathOperator{\Ex}{Ex}
\DeclareMathOperator{\Deficit}{Def}

\newcommand{\eps}{\varepsilon}
\newcommand{\calP}{\mathcal{P}}
\newcommand{\calA}{\mathcal{A}}
\newcommand{\calF}{\mathcal{F}}
\newcommand{\low}{\mathrm{low}}
\newcommand{\lb}{\mathrm{lb}}
\newcommand{\defeq}{\coloneqq}

\newcommand{\varX}{{\textcolor{OliveGreen}X}}
\newcommand{\varF}{{\textcolor{RedOrange}F}}
\newcommand{\varK}{{\textcolor{blue!70}K}}

\DeclareUnicodeCharacter{9001}{\langle}
\DeclareUnicodeCharacter{9002}{\rangle}

% expectation, probability, variance
\newcommand{\Vsymb}{Var}

% better vector definition and some variations
\newcommand{\dir}[1]{{\bm{#1}}}


\long\def\algobox#1{\smallskip
  \noindent
~~\hbox{\fbox{\parbox[c]{0.30\textwidth}{#1}}}
%\smallskip
}

\begin{document}
\begin{multicols}{3}

\title{A Simpler and Faster Strongly}
\title{Polynomial Algorithm for Generalized}
\title{Flow Maximization}
\author{Neil Olver, Laszló A. Végh}
\presenter{Martin B\"ohm}
\centerline{\textit{Doctoral Seminar of Combinatorics, Prague, Fall 2017/2018}}

\section{Generalized Flows}
A directed graph $G$ with \emph{sink} $t$ and:
\begin{itemize}
\item \emph{Demands} on vertices $b_v ∈ ℝ$ (positive or negative)
\item \emph{Gains} on edges $γ_{ij} ∈ ℝ^+$ -- if a unit flow leaves
from $i$ through $ij$, $γ_{ij}$ flow arrives in $j$.
\end{itemize}

\textbf{Goal:} Maximization of incoming flow into $t$.

\thm[Main Theorem]{There exists a strongly polynomial algorithm for
\textsc{Generalized Flow Maximization} with running time $O((m+n\log n) mn \log(n^2/m))$.
}

\section{Basic notation}

$δ^-(S) ≡ $ outedges, $δ^+(S) ≡$ inedges.

Sources and leaks:
	
\[\VN := \{ i \in \Vm : b_i < 0\},\]
\[\VZ := \{ i \in \Vm : b_i = 0\} \]
\[\VP := \{ i \in \Vm : b_i > 0\}.\]

\emph{Net flow:} what ``stays'' at a vertex, i.e.,
\[ \netflow{f}{i} := \sum_{e\in \delta^-(i)} \gamma_e f_e - \sum_{e\in \delta^+(i)} f_e. \]

\emph{Residual edge set:} $\obe{E} ≡$ $\{ij,ji | ij ∈ E\}$.
\section{LP formulation}

\textbf{LP formulation of generalized flows:}

\begin{equation}\tag{P}\label{primal}
\begin{aligned}
 \max \quad & \netflow{f}{t} \\
 \text{s.t.}\quad \netflow{f}{i}&\geq b_i\quad \forall i\in\Vm\\
  f &\geq 0.
\end{aligned}
\end{equation}

\textbf{Dual problem} (linear variables $y_e$ correspond to $\frac{μ_t}{μ_e}$):

\begin{equation}\tag{D}\label{dual}
\begin{aligned}
\max\quad & \mu_t\sum_{j\in \Vm}\frac{b_j}{\mu_j}\\
 \text{s.t.}\quad \mu_j&\ge \gamma_{ij} \mu_i \quad \forall  ij \in E\\
  \mu_t&\in \R^+\\
 \mu_i&\in\R^{+}_{∞}\quad \forall i\in\Vm.
\end{aligned}
\end{equation}

\textbf{Interpretation:} Find minimum \emph{relabelling} such that $\frac{μ_j}{μ_i} ≥ γ_{ij}$.

\section{Feasible labellings}

If $μ$ is a feasible labelling for (D), we can rescale the instance to get

\[ 
    \gamma_{ij}^\mu := \gamma_{ij} \cdot \frac{\mu_i}{\mu_j}, 
        \qquad \text{and} \qquad 
    b_i^\mu:= \frac{b_i}{\mu_i},
\]

and if we have any flow $f$ for (P), we can rescale it to get

\[
    f^\mu_{ij} := \frac{f_{ij}}{\mu_i}
    \qquad \text{and} \qquad 
    \netflow{f^\mu}{i}:=\frac{\netflow{f}{i}}{\mu_i}. 
\]

On the equivalent instance, we have:

\obs[Dual feasibility check]{If $γ_{ij}^μ ≤ 1$, then $μ$ is feasible for (D).
}

\dfn[Tight edge]{
\begin{itemize}
\item An edge is \emph{tight} if $γ_{e}^μ = 1$.
\item All tight edges of $\mu$: $E^μ$.
\end{itemize}
}

\dfn[Excess]{For a flow $f^μ$, we define \emph{excess} of a vertex $v$ to be
$∇f_v^μ - b_v^μ$.
}

\obs[Primal feasibility check]{If all vertices have excess $≥ 0$, $f$ is feasible for (P).
}

\dfn[Non-negative excess and deficit]{
\begin{align*}
\Ex(f,\mu)&:=\sum_{i\in \Vm}\max\left\{\netflow{f^\mu}{i}-b_i^\mu,0\right\} \\
\Deficit(f,\mu)&:=\sum_{i\in   \Vm}\max\left\{b_i^\mu -\netflow{f^\mu}{i},0\right\}.
\end{align*}
}

Relaxed excess to use in the analysis:

\begin{equation*}\globpot(f,\mu) := \sum_{i \in \Vm}
\max\{\netflow{f^\mu}{i}-b^\mu_i, 2\}.
\end{equation*}

\section{Fitting pairs}

\textbf{Complementary slackness:} For any optimal primal and dual solutions $(f^*,μ^*)$, it holds:
$f_e > 0 ⇒ γ_e^μ =1$ and $γ_e^μ < 1 ⇒ f_e = 0$.

\dfn[Fitting pair]{A pair of vectors $(f, μ)$ is a \emph{fitting pair} $≡$ $μ$ is feasible for (D)
and $∀e: f_e > 0 ⇒ γ_{e}^μ =1$.
}

\textbf{Imporant notes:}
\begin{itemize}
\item $f$ does not need to be feasible for (P),
\item \dots or even any sort of a flow (by definition).
\item Though for any fitting pair, $f^μ$ \emph{is} a (regular) flow.
\end{itemize}

\obs{Let $(f,μ)$ be a fitting pair such that $∇f_i = b_i$ for all $i ∈
V ∖ \{t\}$. Then $(f,μ)$ is a pair of primal and dual optimal
solutions.}

\dfn[Safe $μ$]{A relabelling $μ$ is \emph{safe} $≡ ∃f, f$ primal feasible,
and $(f,μ)$ is a fitting pair.}

\lem{$μ$ is safe $⇔$ for every $X ⊆ V ∖ \{t\}$ with $δ^-(X) ∩ E^μ = ∅$
it holds that $b_i^μ(X) ≤ 0$.
}

\section{Algorithm overview}

\textbf{Assumptions of the algorithm:}

\begin{enumerate}
\item An initial fitting pair $(\bar f,\bar \mu)$ is given, where $\bar f\in \R_{+}^E$ is feasible to
\eqref{primal} and $\bar\mu\in (\R^+_0)^V$ is feasible to \eqref{dual}.
\item There is a directed path in $E$ from $i$ to $t$ for every $i\in V$.
\end{enumerate}

\textbf{Simplex trick:} To find an initial fitting pair feasible to
both, create a helper instance (with a trivial feasible initial pair),
run algorithm on it, with the optimum being an initial fitting pair
feasible to both.

\medskip\hrule
Algorithm \textsc{Maximum Generalized Flow}:
\smallskip\hrule\smallskip
\begin{algorithmic}[1]
        \State $\Delta \gets \max_{i \in \Vm} \netflow{\bar
          f^{\bar\mu}}{i}-b_i^{\bar{\mu}}$; $\mu \gets \bar{\mu}\Delta$.\label{alg:delta}
        \State $f \gets $\Call{Round}{$\bar{f}, \mu$}.
        \While{$\VN\cup \VP\neq\emptyset$}
            \State $(f,\mu)\gets $\Call{Produce-Plentiful-Node}{$f,\mu$}.
            \State $(\instance, f,\mu)\gets $\Call{Contract}{$\instance, f,\mu$}. 
              \EndWhile
        \State $\mu\gets$\Call{Expand-to-Original}{$\mu$}
        \State $f\gets$ \Call{Compute-Primal}{$\mu$}
        \State \Return $(f,\mu)$.
\end{algorithmic}

\textbf{Key ideas:}

\begin{enumerate}
\item Maintain a feasible pair $(f,μ)$ at all times such that $f^μ$ is integral (but possibly not feasible).
\item Initial rounding ensures that $b_i^μ -1 ≤ ∇f_i^μ ≤ b_i^μ + 2$ and we maintain this.
\item Find edges that are \emph{contractible} -- tight in all optimal dual solutions.  
\item Compute the optimum only after all contractible edges are found and dual optimal
solution is found. Use alg for regular flows.
\end{enumerate}

\section{Plentiful nodes}

\dfn[Plentiful]{A vertex/node $v$ is \emph{plentiful} with respect to $(f,μ)$ if $|b_v^μ| ≥ 3n(\deg(v) + 1)$.
}

\thm[Plentiful vertex means a contractible edge]{Let $f\in (\R^+_0)^E$ and $\mu\in (\R^{+})^V$, such that $(f,\mu)$ is a
fitting pair with $f^\mu \in \Z_+^E$, $\mu$ is safe, and $\globpot(f,\mu)<2n$. Assume further that
there exists a plentiful node $i$.
Then there exists a contractible arc $e$ incident to $i$, and it can be found in strongly polynomial time.
}

\lem{
Let $(g,\mu)$ be a fitting pair with $\mu$ being safe.
%, and let $\Deficit(g,\mu)=\sum_{i\in
 % \Vm}\max\{b_i^\mu -\netflow{g^\mu}{i},0\}$ denote the total
%violation of node demands.
If $g^\mu_e > \Ex(g, \mu) + \Deficit(g, \mu)$ for an arc $e\in E$,
%then $e$ is tight for every optimal solution $\mu^*$ to \eqref{dual}.
then $e$ is contractible.
}



\section{Core of the algorithm}

\medskip\hrule
Algorithm \textsc{Produce-Plentiful-Node} \textrm{(big picture)}
\smallskip\hrule\smallskip

    \begin{algorithmic}[1]
        \State Maintain that $f^μ$ is integral and $∇f_i^μ$ close to $b_i^μ$.
        \While{there is a node $i$ with $∇f_i^μ ≥ b_i^μ +1$}
            \State Send flow from $i$ through tight edges of $μ$ to a deficient vertex.
            \EndWhile
        \State Update the dual labelling $μ$.
    \end{algorithmic}
% \end{algorithm}

And the actual algorithm for plentiful nodes:

\goodbreak
\medskip\hrule
Algorithm \textsc{Produce-Plentiful-Node}:
\smallskip\hrule\smallskip
\begin{algorithmic}[1]
        \While{there are no plentiful nodes}	
            \State \emph{Augmentation part of the iteration}
            %\State $Q \gets \{ t \} \cup \{ j \in \Vm : \netflow{f^\mu}{i}<b_i^\mu, \text{ and there is an undirected path from $j$ to $t$ in $E^\mu$}\}$. \label{alg:Qdef}
            \State $Q \gets \{ t \} \cup \{ j \in \Vm : \netflow{f^\mu}{i}<b_i^\mu \}$. \label{alg:Qdef}
            \State $\Ss \gets \{ i \in V: \exists \text{ a tight $i$-$Q$-path in } E_f \}$.			
            \While{there exists a node $i \in \Ss \cap \VN$ with $ \netflow{f^\mu}{i} \geq b_i^\mu+1$}
		\State Augment $f^\mu$ by sending $1$ unit from $i$ to
                a vertex in $Q$ along tight arcs. \label{alg:aug} 
		\State Update $Q$ and $\Ss$.
            \EndWhile \vspace{.3cm}
            \State \emph{Label update part of the iteration}\label{alg:rescale}
            \State $\Sc \gets$ the connected component of $\Ss$ w.r.t.\ $\obe{E^\mu}$ containing $t$. \label{alg:Scomp}
            \State $\alpha_0 \gets \min_{e \in \delta^-(\Sc)} 1/\gamma_e^\mu$. \label{alg:alpha0}
    %        \State \emph{(The following will ensure that $e^\mu_i(f) < 2$ for all $i \in S \cap \VN$ after the scaling step)}
	    \State For each $i \in \Sc \cap \VN$, choose
                $\alpha_i \in \left[ \frac{1 - \nabla f_i^\mu}{-b^\mu_i}, 
                    \frac{2 - \nabla f_i^\mu}{-b^\mu_i}\right)$. \label{alg:bigexcess}
                %\Comment{To maintain $\netflow{f^\mu}{i} < b^\mu_i + 2$ for all $i \in S \cap \VN$.}
%            \State For each $i \in S \cap \VN$, choose $\alpha_i$ so that $e^{\mu/\alpha_i}_i(f/\alpha_i) \in [1, 2)$ %; let $\alpha_i = \infty$ if no such choice exists
            \State For each $i \in \Sc \cap \VP$, choose $\alpha_i \in
                \left[\frac{\plenty{i}}{b^\mu_i},
                \frac{\plenty{i}+1}{b^\mu_i} \right)$. \label{alg:newplentiful}
                %\Comment{Stop when a node in $\VP$ becomes plentiful.}
            \State $\alpha \gets \min\{\alpha_0, \min_{i
                         \in \Sc \cap (\VN \cup \VP)}\alpha_i\}$. 
            
            
	    \State $f_e \gets f_e / \alpha$ for all $e \in E[\Sc]$. \label{alg:frescale}
	    \State $\mu_i \gets \mu_i / \alpha$ for all $i \in \Sc$. \label{alg:muupdate}
%\Else \  \Call{Special-Terminate}{$\mu,S$}.
%\EndIf
	\EndWhile
	\State \Return{$(f, \mu)$.}
    \end{algorithmic}

\section{Correctness}

\textbf{Basic notes:}
\begin{itemize}
\item The goal is to ensure that $f^μ$ is integral and unchanged when rescaling.
\item $Q ≡$ vertices which can accept one unit of flow.
\item $\Bars ≡$ vertices which can move flow to $Q$.
\item No edge coming into $S$ is tight: $α_0 > 1$.
\end{itemize}

\lem[Basic properties]{
The following properties hold in every label update step:
\begin{enumerate}[(i)]
        \setlength{\itemsep}{1pt}
\item\label{item:funchanged}
$f^\mu$ remains unchanged.
%Neil: strengthened statement

\item $\alpha>1$ and finite. %\nnote{if needed can add: and, unless $\delta^-(S) = \emptyset$ and $S \cap (\VP \cup \VN) = \emptyset$, it is finite.}
\item $(f, \mu)$ remains a fitting pair. 
\item
If $i\in (V\setminus \Sc)\cup \VZ$, then $b_i^\mu$ does not change.
\item \label{item:bdec}
If $i\in \Sc \cap \VN$, then $b_i^\mu$ decreases. After the label update,
$\netflow{f^\mu}{i}\le b_i^\mu+2$. 
\item \label{item:binc}
If $i\in \Sc \cap \VP$, then $b_i^\mu$ increases. If $\alpha=\alpha_i$ for
such an $i$, then $\plenty{i}\le b_i^\mu\le \plenty{i}+1$ after the label update.
%\nnote{Changed strictness of plentiful definition, so the non-strictness here should now be fine.}
%\lnote{Added upper and lower bounds.}
\end{enumerate}
}

\lem[Safeness]{
The labelling $\mu$ remains safe throughout the algorithm.
}

\lem[Potential bound]{
$\globpot(f,\mu)< 2n-1$ holds throughout the algorithm \textsc{Contract} and \textsc{Produce-Plentiful-Node}. Consequently,
$\Ex(f,\mu)< 2n-1$ holds throughout.
}

\section{Work per augmentation}

\lem[Between two augmentations]{In algorithm \textsc{Produce-Plentiful-Node}, the algorithm performs $O(n^2)$ operations
between two successive augmentations of the flow.}

\lem[Lower bound on net flow]{
For any $i\in\VN$, 
$b_i^\mu-1< \netflow{f^\mu}{i}$ holds at any point of the algorithm.
Once $b_i^\mu\le \netflow{f^\mu}{i}$ holds in
\textsc{Produce-Plentiful-Node}, this property is maintained until the
end of the subroutine.
}

\dfn[Potentials]{

\emph{Demand potential}: 
\[
%    \Psi(\mu) := \sum_{i\in\VN} (3nd_i+3n+3+b^\mu_i).
    \Psi(\mu) := -\sum_{i\in\VN} b^\mu_i.
\]
\emph{Excess potential}:
\[
	\Phi(f,\mu) := \sum_{i \in \VN} \netflow{f^\mu}{i}-b^\mu_i.
 \]

}

\lem[Demand potential is increasing]{
During \textsc{Produce-Plentiful-Node},
the potential $\Psi(\mu)$ is increasing. 
If $r$ augmentations
are performed, then $\Psi(\mu)$ increases by at least $\min\{r-4n,0\}$.
}

\lem[Bound on demand potential]{
$\Psi(\mu) = O(mn)$ throughout the algorithm.
}

\textbf{Consequence:} $O(mn^2)$ augmentation in one run of \textsc{Produce-Plentiful-Node}.
\vspace{42em}
~ 
\vspace{\textheight}
~
 
\end{multicols}
\end{document}
