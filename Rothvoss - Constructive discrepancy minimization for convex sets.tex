\input macros/cheatmac
%\usepackage{auto-pst-pdf}
%\usepackage{pst-...}

\def\Pee{\mathbb{P}}
\def\Zet{\mathbb{Z}}
\def\Rko{\mathbb{R}}
\def\Complex{\mathbb{C}}
\def\Dis{\mathcal{D}}

\def\Bigpi{{\rm Par}}
\def\intcone{{\rm intcone}}
\def\cone{{\rm cone}}
\def\conv{{\rm conv}}
\def\vert{{\rm vert}}
\def\Las{{\textsc{Las}}}
\def\Oh{{\rm O}}
\def\supp{{\rm supp}}
\def\enc{{\rm enc}}
\newcommand{\eps}{\varepsilon}
\let\cfix=\cdot

% expectation, probability, variance
\newcommand{\Esymb}{\mathbb{E}}
\newcommand{\Psymb}{\mathbb{P}}
\newcommand{\Vsymb}{Var}

% better vector definition and some variations
\renewcommand{\vec}[1]{{\bm{#1}}}
\newcommand{\bvec}[1]{\bar{\vec{#1}}}
\newcommand{\pvec}[1]{\vec{#1}'}
\newcommand{\tvec}[1]{{\tilde{\vec{#1}}}}



\DeclareMathOperator*{\E}{\Esymb}
\DeclareMathOperator*{\Var}{\Vsymb}
\DeclareMathOperator*{\ProbOp}{\Psymb}

\renewcommand{\Pr}{\ProbOp}

\newcommand{\problemmacro}[1]{\textsc{#1}\xspace}
\newcommand{\maxtwocsp}{\problemmacro{Max 2-Csp}}
\newcommand{\uniquegames}{\problemmacro{Unique Games}}

\newcommand{\norm}[1]{\lVert#1\rVert}
\newcommand{\Norm}[1]{\left\lVert#1\right\rVert}
\newcommand{\bignorm}[1]{\big\lVert#1\big\rVert}
\newcommand{\Bignorm}[1]{\Big\lVert#1\Big\rVert}

\newcommand{\normo}[1]{\norm{#1}_1}
\newcommand{\Normo}[1]{\Norm{#1}_1}
\newcommand{\bignormo}[1]{\bignorm{#1}_1}
\newcommand{\Bignormo}[1]{\Bignorm{#1}_1}

\newcommand{\set}[1]{\{#1\}}
\newcommand{\Set}[1]{\left\{#1\right\}}
\newcommand{\bigset}[1]{\big\{#1\big\}}
\newcommand{\Bigset}[1]{\Big\{#1\Big\}}

% absolute value
\newcommand{\abs}[1]{\lvert#1\rvert}
\newcommand{\Abs}[1]{\left\lvert#1\right\rvert}
\newcommand{\bigabs}[1]{\big\lvert#1\big\rvert}
\newcommand{\Bigabs}[1]{\Big\lvert#1\Big\rvert}

% square brackets
\newcommand{\brac}[1]{[#1]}
\newcommand{\Brac}[1]{\left[#1\right]}
\newcommand{\bigbrac}[1]{\big[#1\big]}
\newcommand{\Bigbrac}[1]{\Big[#1\Big]}

\newcommand{\ia}{{ia}}
\newcommand{\jb}{{jb}}
\newcommand{\mper}{\,.}
% \newcommand{\Mid}{\;\middle\vert\;}

\newcommand{\iprod}[1]{\langle#1\rangle}
\newcommand{\Iprod}[1]{\left\langle#1\right\rangle}

\newcommand{\paren}[1]{(#1)}
\newcommand{\Paren}[1]{\left(#1\right)}
\newcommand{\bigparen}[1]{\big(#1\big)}
\newcommand{\Bigparen}[1]{\Big(#1\Big)}

\newcommand{\Ins}{\mathcal{I}}
\newcommand{\El}{\mathcal{L}}
\newcommand{\F}{\mathcal{F}}
\newcommand{\defeq}{\coloneqq}
\newcommand{\rank}{\textrm{rank}}
\newcommand{\psdrank}{\textrm{rk}_{\textrm{psd}}}
\newcommand{\degsos}{\textrm{deg}_{\textrm{sos}}}
\newcommand{\detlb}{\textrm{detlb}}
\newcommand{\disc}{\textrm{disc}}
\newcommand{\herdisc}{\textrm{herdisc}}
\newcommand{\Corr}{\textsc{Corr}}
\newcommand{\Splus}{\mathcal{S}_+}
\newcommand{\Tr}{\textrm{Tr}}
\newcommand{\scalar}[2]{\langle #1, #2 \rangle}
\newcommand{\vectext}{\textrm{vec}}
\DeclareMathOperator{\val}{val}
\DeclareMathOperator*{\Cov}{Cov}


\long\def\algobox#1{\smallskip
  \noindent
~~\hbox{\fbox{\parbox[c]{0.30\textwidth}{#1}}}
%\smallskip
}

% Rothvoss

\providecommand{\setR}{\mathbb{R}}

\begin{document}
\begin{multicols}{3}

\title{Constructive Discrepancy Minimization}
\title{for Convex Sets}
\author{Thomas Rothvoss}
\presenter{Martin B\"ohm}
\centerline{\textit{Combinatorial PhD Seminar, Fall 2015/2016}}

%TODO: \Pee, \Zet, \Rko
\section{Discrepancy}

\dfn[Discrepancy]{We work in a universe $U = [n]$, and we are given a list of sets $\F$, where each $f ∈ \F$
is a subset of $U$. Our goal is to color the universe by two colors so that all sets in $\F$ are as balanced
as possible.

Formally, given a coloring $x∈\{-1,1\}^n$, we have $\disc(\F,x) \defeq \max_{F ∈ \F} |∑_{i ∈ F} x_i|$ and
our minimization goal $\disc(\F) = \min_{x ∈ \{-1,1\}^n} \disc(\F,x)$.
}

\noindent\textbf{Problem:} Assuming $P≠NP$ and assuming that $m =
O(n)$, we cannot distinguish between $\F$ with discrepancy $0$ and
$\F$ with discrepancy $\sqrt{n}$.

\dfn[Hereditary discrepancy]{ $\herdisc(\F) = \max_{J ⊆ U} \disc(\F|_{J})$. }
\section{Related work}

\thm[Spencer, 1985]{Given a set system with $n$ elements and $n$ sets, there
exists a coloring of discrepancy $O(\sqrt{n})$.
}

\thm[Bansal, 2010]{
\begin{enumerate}
\item Using an SDP and a random walk guided by it, one can find the coloring from Spencer's theorem in polynomial time.
\item There is a randomized polytime algorithm which for a set system $\F$ with $\herdisc(\F) \leq H$ computes a coloring with discrepancy $O(H \log (mn))$.
\end{enumerate}
}

\thm[Lovett, Meka, 2012]{
\begin{enumerate}
\item Using just a random walk, one can find the coloring from Spencer's theorem in polynomial time. 
\item Given a symmetric bounded convex polytope thta is ``large enough'', one can find in polynomial time a point $y$ that has at least half of the coordinates in $\{-1,1\}$.
\end{enumerate}
}

\thm[Matousek, Nikolov, Talwar 2015]{An $O(\log |\F|)$-approximation for $herdisc(\F)$ using
an SDP program computing a special norm.
}

\noindent\textbf{Afterwards/independently:}

\thm[Eldan, Singh, 2014]{
\begin{itemize}
\item The following algorithm is equivalent to the algorithm in the main theorem:
\begin{enumerate}
\item[(1)] Take a uniform random direction $c$.
\item[(2)] Optimize the program $\max c^Tx; x \in K \cap [-1,1]^n$.
\end{enumerate}
\item The condition of symmetry on $K$ in the main theorem can be dropped.
\end{itemize}
}



\section{Main theorem}

\thm[Main theorem]{
There is a randomized polynomial time algorithm, which for any
symmetric convex set $K \subseteq \setR^n$ with Gaussian measure at
least $e^{-n/500}$ finds a point $y \in K \cap [-1,1]^n$ with $y_i \in
\{ - 1,1\}$ for at least $\frac{n}{9000}$ many coordinates. Here it
suffices if a polynomial time separation oracle for the set $K$
exists.
}

\begin{center}
{\bf Algorithm: }
\begin{enumerate}
\item[(1)] take a random Gaussian vector $x^* \sim N^n(0,1)$
\item[(2)] compute the point \\ $y^* = \textrm{argmin}\{ \| x^* - y\|_2 \mid y \in K \cap [-1,1]^n\}$
\item[(3)] return $y^*$
\end{enumerate}
\end{center}

\centerline{\includegraphics[scale=0.6]{rothvoss14-picture.png}}

\section{Tools}
\dfn[symmetry of convex sets]{$x \in K \iff -x \in K$
}

\dfn[unit coefficients]{$y_i$ unit coefficient $\iff y_i \in \{-1,1\}.$
}

\dfn[$n$-dimensional Gaussian distribution]{
$\gamma_n \equiv$ measure with density$\frac{1}{(2\pi)^{n/2}} e^{-\|x\|_2^2/2}$ for $x \in \setR^n.$ 
$\gamma_n(K) = \Pr_{x \sim N^n(0,1)}[x \in K]$ whenever $K$ is a measurable set. 
}

\dfn[strip]{$S_\lambda$ strip with \emph{width} $2\lambda$ $\equiv S_\lambda = \{x \in \setR^n \mid |\langle v,x\rangle| \leq \lambda \}$.
}

\dfn[binary entropy function]{$H(\varepsilon) = \varepsilon \log_2 1/\varepsilon + (1-\varepsilon) \log_2 1/(1-\varepsilon)$. Fact: for $0 < \varepsilon \leq 1/2$, ${n choose \varepsilon n} \le H(\varepsilon) \leq e^{1.5 \varepsilon \log_2 1/\varepsilon}$.
}

\thm[Isoperimetric ineq.]{
Let $K \subseteq \setR^n$ be a measurable set and $H$ be a halfspace so that $\gamma_n(K) = \gamma_n(H)$.
Then for any $\delta \geq 0$, $\gamma_n(K_{\delta}) \geq \gamma_n(H_{\delta})$.
}

\lem[Measure concentration for distance]{Let $\alpha > 0$. Then for any measurable set $K$ with $\gamma_n(K) \geq e^{-\alpha n}$ one
has $\gamma_n(K_{3\sqrt{\alpha n}}) \geq 1 - e^{-\alpha n}$.
}

\lem[Sidak, Khatri]{
Let $K \subseteq \setR^n$ be a symmetric convex body and $S \subseteq \setR^n$ be a strip. 
Then $\gamma_n(K \cap S) \geq \gamma_n(K) \cdot \gamma_n(S)$.
}

\lem[Some calculations]{
\begin{enumerate}
\item $\gamma_n(S_1) \geq e^{-1/2}$.
\item $\gamma_n(S_\lambda) \geq 1 - e^{-\lambda^2 / 2}$.
\item $2 \gamma_1([2,\infty)) \geq 1/25$.
\end{enumerate}
}

\section{Main proof}
\thm[Main theorem again]{
Let $0 < \varepsilon \leq \frac{1}{9000}$ be a constant and $\delta :=
\frac{3}{2} \varepsilon \log_2(\frac{1}{\varepsilon})$. Suppose that
$K \subseteq \setR^n$ is a symmetric, convex body with $\gamma_n(K)
\geq e^{-\delta n}$. Choose a random Gaussian $x^* \sim N^n(0,1)$ and
let $y^*$ be the point in $K \cap [-1,1]^n$ that minimizes $\|x^* -
y^*\|_2$. Then with probability $1-e^{-\Omega(n)}$, $y^*$ has at least
$\varepsilon n$ many coordinates $i$ with $y_i^* \in \{ -1,1\}$.
}

\medskip\hrule\smallskip
Assume less than $\varepsilon n$ unit coefficients in $y^*$.
Two main conflicting facts:

\begin{enumerate}
\item $x^*$ is far away from $[-1,1]^n$ -- $d(x^*, [-1,1]^n) \geq \frac{\sqrt{n}}{5}$.
\item Due to measure concentration:
\[ d( x^*, K \cap \{x \mid \left|x_i\right| \leq 1 \text{ for every unit } i \text{ in } y^{*}\} ) < \frac{\sqrt{n}}{5} \]
\end{enumerate}

However, both distances are equal!

\section{Four steps to Spencer}

\thm[Step 1 -- subspace]{
Fix $0 < \varepsilon \leq \frac{1}{60000}$ and $\delta := \frac{3}{2} \varepsilon \log_2(\frac{1}{\varepsilon})$. 
Let $K \subseteq \setR^n$ be a symmetric, convex body 
with $K \subseteq H$ and $\gamma_H(K) \geq e^{-\delta n}$
where $H = \{ x \in \setR^n \mid \left<v_i,x\right>=0 \; \forall i\in [m] \}$
is a subspace defined by $m \leq 2\delta n$ equations.
Choose a random Gaussian $x^* \sim N^n(0,1)$ and let $y^*$ be the point in $K \cap [-1,1]^n$ that
minimizes $\|x^* - y^*\|_2$.
Then with probability $1-e^{-\Omega(n)}$, $y^*$ has at least $\varepsilon n$
many coordinates $i$ with $y_i^* \in \{ -1,1\}$.
}

\thm[Step 2 -- translation]{
Let $\varepsilon \leq \frac{1}{60000}$ and $\delta := \frac{3}{2}\varepsilon \log_2(\frac{1}{\varepsilon})$.

Given a subspace $H \subseteq \setR^n$ of dimension at least $(1-\delta)n$, 
a symmetric convex set $K \subseteq H$ with $\gamma_H(K) \geq e^{-\delta n}$ 
and a point $c \in {\left]-1,1\right[}^n$. There exists a polynomial time algorithm 
to find a point $y \in (c + K) \cap [-1,1]^n$ so that at least 
$\frac{\varepsilon}{2} n$ many indices $i$ have $y_i \in \{ -1,1\}$.
}

\thm[Step 3 -- blowup]{
Suppose that $K \subseteq \setR^n$ is a symmetric convex body so that for all
axis-parallel subspaces $U \subseteq \setR^n$ %  $H := \{ x \mid x_i=0 \forall i \in J\}$ with $J \subseteq [n]$
one has that $\gamma_U(K) \geq e^{-\dim(U)/500}$. Then there is a polynomial time algorithm
to compute a $y \in \{ \pm 1\}^n \cap O(\log n) \cdot K$.
}

\cor[Step 4 -- constant blowup]{
Suppose that $K \subseteq \setR^n$ is a symmetric convex body so that
for all axis parallel subspaces $U \subseteq \setR^n$ one has
$\gamma_U((\frac{\textrm{dim}(U)}{n})^{\varepsilon} K) \geq e^{-\dim(U)/500}$ for some constant $\varepsilon > 0$. 
Then one can compute a vector $y \in \{ \pm 1\}^n \cap (c_{\varepsilon} K)$ in polynomial time.
}

\centerline{\includegraphics[scale=0.35]{rothvoss14-simplecalc.png}}
\end{multicols}
\end{document}
