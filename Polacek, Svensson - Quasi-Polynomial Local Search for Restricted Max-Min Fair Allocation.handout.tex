\input cheatmac

\font\tenscr=rsfs10   % The names \tenscr, \sevenscr, \fivescr can
\font\sevenscr=rsfs7  % be any command names, but should be unique and
\font\fivescr=rsfs5   % descriptive.

% Allocate the math family...
\newfam\scrfam        % The name \scrfam can be any command, but should
                      % be unique and descriptive.
% ...and assign the fonts:
\textfont\scrfam=\tenscr
\scriptfont\scrfam=\sevenscr
\scriptscriptfont\scrfam=\fivescr

% I don't know if this is correct or necessary:
\skewchar\tenscr=127
\skewchar\sevenscr=127
\skewchar\fivescr=127

% Define a command to switch to this new math family:
\def\scr{\fam\scrfam}
%%% end

\def\hyp#1{{\cal #1}}
\def\ind{{\cal I}}

\def\author#1{\centerline{\bf #1}}
\def\presenter#1{\centerline{\it Presented by #1}}

\title{Quasi-Polynomial Local Search}
\title{for Restricted Max-Min Fair Allocation}
\author{Lukáš Poláček, Ola Svensson}
\presenter{Martin Böhm}
\centerline{{\it Approximation and Online Algorithm Seminar, MFF UK}}

% \section{Definitions and notation}

\dfn[Max-min fair allocation]{Given the set of $\hyp{R}$ resources and
$\hyp{P}$ players, find an allocation of resources to the players such
that the least satisfied player is satisfied as much as possible.
Also known as the {\it Santa Claus problem.}
}

\dfn[Restricted max-min fair allocation]{In our setting, while
resources have different values to all players ($v_{ij}$ for the
resource $i$-th player and $j$-th resource) there are only two
possible values that can arise for a given player: $v_{ij} = 0$ or
$v_{ij} = v_J$.
}

\thm[Main theorem]{For any $ε∈(0,1]$ we can find a ${1 \over 4+ε}$-approximation
algoritthm for restricted max-min fair allocation in time $n^{O({1 \over ε} log n)}$, where
$n = |\hyp{P}| + |\hyp{R}|$.
}

\obs{Our algorithm tries to do allocation for a given $T$ which is
said to be the optimum solution. If $T$ is not a feasible solution, it
fails and we can use this to apply a binary search on the optimal
value of $T$.

Therefore, for the rest of the talk, we assume $T$ is
fixed and it is the optimum solution of the restricted max-min fair
allocation problem. For the simplicity of the argument, we also assume
$T = 1$ through scaling of the values on input.}

\section{Configuration LP}

\dfn[Configuration LP]{
The configuration LP has exponentially many conditions, based on every
admissible assignment of the resources to a player. More formally, it is
the following linear program $CLP(T)$ parametrized by $T$:

$$ \min 0,$$
$$∀i ∈ \hyp{P}: ∑_{C ∈ \hyp{C}(i,T)} x_{i,C} ≥ 1, $$
$$∀j ∈ \hyp{R}: ∑_{i,C: j ∈ C, C ∈ \hyp{C}(i,T)} x_{i,C} ≤ 1, $$
$$ x ≥ 0. $$
}

\dfn{ The dual of the Configuration LP can be
stated as:

$$ \max ∑_{i ∈ \hyp{P}} y_i  - ∑_{j ∈ \hyp{R}} z_j,$$
$$ ∀i ∈ \hyp{P}, ∀C ∈ C(i,T): y_i ≤ ∑_{j ∈ C}z_j, $$
$$ y,z ≥ 0. $$
}

\section{Local search}

We will solve the problem by using a local search algorithm for a special
kind of hypergraph matching. $α>4$ will be our approximation parameter.

\dfn{The ground set of our hypergraph will be $\hyp{P} ∪ \hyp{R}$.
A hyperedge $e$ in our setting will contain a player $e_P$ along with 
an inclusion-minimal set of resources such that $e_P$ has assigned value
at least $1/α$.}

\dfn[Thin, fat edges]{A {\it fat} edge will be an edge such that it contains only one resource (of size at
least $1/α$. A {\it thin} edge is any non-fat edge.}

\dfn[Length]{The {\it length} of a fat edge will be set as 0, of any thin edge as 1.}

In our scenario, we will try to build an {\it alternating path tree}
on our hypergraph until we hit a certain depth. Suppose that we have
already created a partial matching $M$. The tree is contains two
kinds of edges. The first group are the edges we have decided to try to
add (denoted as $A$) and the other group are the edges in $M$ which block
the addition of the edges in $A$ -- these will be denoted $B$.

\dfn[Addable edge]{An edge $e$ is {\it addable} to the alternating path tree if it is disjoint from 
any edge in $A ∪ B$ and it contains a player that is already in our tree.
}

\dfn[Blocking edge]{An edge $b$ in the matching $M$ is {\it blocking} an edge $e$ if it intersects the
edge in a resource.
}

\dfn[Layers]{For the sake of the analysis, we decompose the edge sets $A$ and $B$ into classes based on the
distance to the root, denoting them as $A_0$, $B_0$, $A_1$, $B_1$ etc. We also denote $A_i^t$ ($B_i^t$) to be the
subset of edges in $A_i$ ($B_i$) which are thin, and similarly $A_i^f$ ($B_i^f$) for fat edges.}

\subsection{Local search algorithm}

Input: a partial matching $M$.

Output: an increased matching, provided $T$ is a feasible solution.

\numlist\nnorm
\: Find an unmatched player, make it a root of the tree.
\: While an addable edge is within distance $2 \log_{(α-1)/3}(|\hyp{P}|)+1$: 
  \: \hskip 0.5cm Find an addable edge $e$ of minimum distance from the root.
  \: \hskip 0.5cm Add $e$ to the tree.
  \: \hskip 0.5cm If it has blocking edges, add them to the tree also.
  \: \hskip 0.5cm While $e$ has no blocking edges:
      \: \hskip 1cm Add $e$ to the matching $M$.
      \: \hskip 1cm Remove its parent edge $b$ from the matching $M$.
      \: \hskip 1cm Repeat the procedure for the grandparent of $e$.
  \: \hskip 0.5cm Remove all edges that are in the same or greater distance as the last edge added
to the matching.
\endlist

\section{Analysis}

\obs{The algorithm does go through all fat edges first, selecting an edge of distance $1$ only after
the subtree of distance $0$ has been traversed.
}

\lem[Runtime]{For a desired approximation guarantee of $1/α = 1/(4+ε)$, the algorithm
terminates in time $n^{O({1\over ε}\log n)}$.
}

\prf{Create a signature vector:
$$(-|A_0|, |B_0|, - |A_1|, |B_1|, … , -|A_{2k}|, |B_{2k}|, ∞).$$

We prove that any operation decreases the lexicographic size of the vector. Since
the length of the vector is bounded by the limit in the algorithm, we get the desired
runtime bound.
}

\lem[Key lemma]{ Let $α > 4$. Assuming that $CLP(T)$ is feasible, if there is no addable
edge within distance $2D+1$ from the root for some integer $D$, then
$${(α-4) \over 3} ∑_{i=1}^D |B_{2i}^t| < |B_{2D+2}^t|.$$
}

\cor[Correctness]{If $α>4$ and $CLP(T)$ is feasible, there always is an addable edge within
distance $2D+1$ for $D = \log_{(α-1)/3}(|\hyp{P}|)$. 
}

\obs{Every blocking edge can be mapped to exactly one vertex of $\hyp{P}$.}

\obs{The size of each thin edge is at most $2/α$.}

\obs{Any part of a $B$-edge that is not contained in any $A$-edge must
be of size $1/α$.}

\bye