\input macros/cheatmac

\usepackage{bm} % bold math

% left align
\usepackage{environ}
\makeatletter
\NewEnviron{Lalign}{\tagsleft@true\begin{alignat*}{1}\BODY\end{alignat*}}
\makeatother

% small spacing in align
% \abovedisplayskip=0pt plus 3pt
% \abovedisplayshortskip=0pt plus 3pt
% \belowdisplayskip=0pt plus 3pt 
% \belowdisplayshortskip=0pt plus 3pt

\def\Pee{\mathbb{P}}
\def\Zet{\mathbb{Z}}
\def\Rko{\mathbb{R}}
\def\Complex{\mathbb{C}}
\def\Dis{\mathcal{D}}

\def\Bigpi{{\rm Par}}
\def\intcone{{\rm intcone}}
\def\cone{{\rm cone}}
\def\conv{{\rm conv}}
\def\vert{{\rm vert}}
\def\Las{{\textsc{Las}}}
\def\Oh{{\rm O}}
\def\supp{{\rm supp}}
\def\enc{{\rm enc}}

\newcommand{\eps}{\varepsilon}
\let\cfix=\cdot

% expectation, probability, variance
\newcommand{\Esymb}{\mathbb{E}}
\newcommand{\Psymb}{\mathbb{P}}
\newcommand{\Vsymb}{Var}

% better vector definition and some variations
\renewcommand{\vec}[1]{{\bm{#1}}}
\newcommand{\bvec}[1]{\bar{\vec{#1}}}
\newcommand{\pvec}[1]{\vec{#1}'}
\newcommand{\tvec}[1]{{\tilde{\vec{#1}}}}
\newcommand{\calI}{\mathcal{I}}
\newcommand{\B}{\mathcal{B}}
\DeclareMathOperator*{\E}{\Esymb}
\DeclareMathOperator*{\Var}{\Vsymb}
\DeclareMathOperator*{\ProbOp}{\Psymb}

\renewcommand{\Pr}{\ProbOp}

\newcommand{\problemmacro}[1]{\textsc{#1}\xspace}
\newcommand{\maxtwocsp}{\problemmacro{Max 2-Csp}}
\newcommand{\uniquegames}{\problemmacro{Unique Games}}

\newcommand{\norm}[1]{\lVert#1\rVert}
\newcommand{\Norm}[1]{\left\lVert#1\right\rVert}
\newcommand{\bignorm}[1]{\big\lVert#1\big\rVert}
\newcommand{\Bignorm}[1]{\Big\lVert#1\Big\rVert}

\newcommand{\normo}[1]{\norm{#1}_1}
\newcommand{\Normo}[1]{\Norm{#1}_1}
\newcommand{\bignormo}[1]{\bignorm{#1}_1}
\newcommand{\Bignormo}[1]{\Bignorm{#1}_1}

\newcommand{\set}[1]{\{#1\}}
\newcommand{\Set}[1]{\left\{#1\right\}}
\newcommand{\bigset}[1]{\big\{#1\big\}}
\newcommand{\Bigset}[1]{\Big\{#1\Big\}}

% absolute value
\newcommand{\abs}[1]{\lvert#1\rvert}
\newcommand{\Abs}[1]{\left\lvert#1\right\rvert}
\newcommand{\bigabs}[1]{\big\lvert#1\big\rvert}
\newcommand{\Bigabs}[1]{\Big\lvert#1\Big\rvert}

% square brackets
\newcommand{\brac}[1]{[#1]}
\newcommand{\Brac}[1]{\left[#1\right]}
\newcommand{\bigbrac}[1]{\big[#1\big]}
\newcommand{\Bigbrac}[1]{\Big[#1\Big]}

\newcommand{\ia}{{ia}}
\newcommand{\jb}{{jb}}
\newcommand{\mper}{\,.}
% \newcommand{\Mid}{\;\middle\vert\;}

\newcommand{\iprod}[1]{\langle#1\rangle}
\newcommand{\Iprod}[1]{\left\langle#1\right\rangle}

\newcommand{\paren}[1]{(#1)}
\newcommand{\Paren}[1]{\left(#1\right)}
\newcommand{\bigparen}[1]{\big(#1\big)}
\newcommand{\Bigparen}[1]{\Big(#1\Big)}

\newcommand{\Ins}{\mathcal{I}}
\newcommand{\El}{\mathcal{L}}
\newcommand{\Scal}{\mathcal{S}}
\newcommand{\defeq}{\coloneqq}
\newcommand{\rank}{\textrm{rank}}
\newcommand{\psdrank}{\textrm{rk}_{\textrm{psd}}}
\newcommand{\degsos}{\textrm{deg}_{\textrm{sos}}}
\newcommand{\detlb}{\textrm{detlb}}
\newcommand{\disc}{\textrm{disc}}
\newcommand{\herdisc}{\textrm{herdisc}}
\newcommand{\Corr}{\textsc{Corr}}
\newcommand{\Splus}{\mathcal{S}_+}
\newcommand{\Tr}{\textrm{Tr}}
\newcommand{\scalar}[2]{\langle #1, #2 \rangle}
\newcommand{\vectext}{\textrm{vec}}
\DeclareMathOperator{\val}{val}
\DeclareMathOperator*{\Cov}{Cov}

\newcommand{\Jtop}{J_{\text{top}}}
\newcommand{\Jmid}{J_{\text{middle}}}
\newcommand{\Jbot}{J_{\text{bottom}}}


\long\def\algobox#1{\smallskip
  \noindent
~~\hbox{\fbox{\parbox[c]{0.30\textwidth}{#1}}}
%\smallskip
}

\begin{document}
\begin{multicols}{3}

\title{An Algorithm for Komlós Conjecture}
\title{Matching Banaszczyk's Bound}
\author{Nikhil Bansal, Daniel Dadush, Shashwat Garg}
\presenter{Martin B\"ohm}
\centerline{\textit{Approximation and Online Seminar, Prague, Fall 2016/2017}}

\section{Discrepancy}

\dfn[Discrepancy]{We work in a universe $U = [n]$, and we are given a list of sets $\Scal$, where each $f ∈ \Scal$
is a subset of $U$. Our goal is to color the universe by two colors so that all sets in $\Scal$ are as balanced
as possible. Formally: given a coloring $x∈\{-1,1\}^n$, we have $\disc(\Scal,x) \defeq \max_{F ∈ \Scal} |∑_{i ∈ F} x_i|$ and
our goal is $\disc(\Scal) = \min_{x ∈ \{-1,1\}^n} \disc(\Scal,x)$.
}

\dfn[Beck-Fiala setting]{ In the \emph{Beck-Fiala setting}, we have an additional constraint
that the number of occurences of every element $i ∈ U$ is limited by $t$, i.e. $|\{F ∈ \Scal \mid i ∈ F\}| ≤ t$.
}

\thm[Beck-Fiala]{For any family $\Scal$ in the Beck-Fiala setting, it holds that $\disc(\Scal) ≤ 2t-1$.
Proven via an iterative rounding algorithm.
}

\opn[Beck-Fiala]{In the Beck-Fiala setting, it could hold that $\disc(\Scal) ≤ \sqrt{t}$.
}

\thm[Banaszczyk]{$\disc(\Scal) ≤ \sqrt{t · \Oh(\log n)}$. Proven non-constructively using convex geometrical
arguments. Rumored to be very difficult to get into.
}

\thm[B-D-G]{$\disc(\Scal) ≤ \sqrt{t · \Oh(\log n)}$ constructively using an SDP iterative rounding.
}

\section{Notation}
$x_k\in [-1,1]^n ≡ $ coloring at the {\em end} of time step $k$. 

\begin{itemize}
\item Frozen variables $≡$ set to at least $(1-1/n)$ in abs. value.
\item Active $≡$ others.
\item $A_k ≡$ active vars at the start of the time step.
\end{itemize}

\begin{itemize}
\item $S ∈ \Scal$ \emph{big} at time $k$ $≡$ $≥ at$ variables alive at that time.
\item $\B_k ≡ $ big sets; $\mathcal{L}_k ≡$ little (non-big) sets.
\end{itemize}
\section{The Algorithm}
Constants: $a = 6$, $γ ≡ 1/(n^2 \log n)$, $T ≡ (12/γ^2) \log n$.
\begin{enumerate}
\item Initialize $x_0(i) =0$ for all $i\in [n]$ and $A_1=\{1,2,...,n\}$. 
%Let $\gamma$ be parameters to be defined later.
\item  For each time step $k=1,2,\ldots, T$ repeat the following:
\begin{enumerate}
%\item Remove all sets $S_j$ with $|S_j\cap A(k-1)|\le 25\sqrt{t\,log\,n}$.
\item \label{apx:step1}
 Find a solution to the following SDP:
\begin{eqnarray}
	\textrm{Maximize }\sum_{i\in A_k} \|u_i\|_2^2 \nonumber\\
	\textrm{s.t.} \qquad%\nonumber\\
	\label{sdp1}  \|\sum_{i \in S\cap A_k} u_i\|_2^2 & = &  \ 0 \qquad  \textrm{for each }  S \in \mathcal{B}_k \nonumber\\
%	\label{sdp2} \|\sum_{i \in A_k} x_{k-1}(i)u_i\|_2^2 & = &  \ 0 \\
	\|\sum_{i \in S\cap A_k} u_i\|_2^2 & \leq &  \ 2\sum_{i\in S\cap A_k}\| u_i\|_2^2 \qquad   \textrm{for }  S \in \mathcal{L}_k \label{sdp3}  \nonumber\\
	\| \sum_{i\in S\cap A_k}x_{k-1}(i) u_i\|_2^2  & \leq & \ 2\sum_{i\in S\cap A_k}\| u_i\|_2^2 \qquad   \textrm{for }  S \in \mathcal{L}_k \qquad \label{sdp4} \nonumber \\
   	\|u_i\|_2^2 & \le &  1  \qquad \forall i \in A_k \nonumber 
\end{eqnarray}
\item 
\label{apx:round}
Let $r_k \in ℝ^n$ be a uniformly random $\pm 1$ vector.
For each $i \in A_k$, update $x_k(i)=x_{k-1}(i)+\gamma\langle r_k, u_i\rangle$. 
For  each $i \not\in A_k$, set $x_k(i)=x_{k-1}(i)$.

\item
\label{apx:rnd}
 Initialize $A_{k+1}=A_{k}$.

For each $i$, if $|x_k(i)|\geq 1-1/n$, update $A_{k+1} = A_{k+1} \setminus \{i\}$.
\end{enumerate}
\item 
\label{stp3}
Generate the final coloring as follows.
For the frozen elements $i\notin A_{T+1}$, set $x_T(i)=1$ if $x_T(i)\geq 1-1/n$ and $x_T(i)=-1$ otherwise.
For the alive elements $i \in A_{T+1}$, % randomly round $x_T(i)$ to $1$  with probability $(1+x_T(i))/2$ and $-1$ otherwise.
set them arbitrarily to $\pm 1$. 
\end{enumerate}

\section{SDP}
\begin{eqnarray*}
	\textrm{Primal (matrix form): Maximize } I\bullet X && \quad \textrm{subject to}  \\
	v_{S}v_{S}^T\bullet X & = &  \ 0 \qquad \textrm{for each }  S \in \mathcal{B}_k \\
	(v_{S}v_{S}^T-2I_{S})\bullet X & \leq &  \ 0 \qquad \textrm{for each }  S \in \mathcal{L}_k \\
	(x_{S}x_{S}^T-2I_{S})\bullet X & \leq &  \ 0 \qquad  \textrm{for each }  S \in \mathcal{L}_k \\
	(e_ie_i^T)\bullet X & \leq &  \ 1 \qquad \forall i \in A_k \nonumber \\
   	X & \succeq & \ 0
\end{eqnarray*}
\begin{eqnarray*}
	\textrm{Dual: Minimize } \sum_{i\in A_k} b_i \quad \textrm{subject to} \\
	\sum_{i\in A_k}  b_i e_ie_i^T  +   \sum_{S \in \mathcal{B}_k}\alpha_{S} v_{S}v_{S}^T \quad + & \\
\qquad + \sum_{S \in \mathcal{L}_k}\left(\beta_{S} (v_{S}v_{S}^T-2I_{S})+\beta_{S}^x(x_{S}x_{S}^T-2I_{S})\right)  &\succeq   \ I \\
	 ∀ i ∈ A_k\colon \qquad b_i  &\geq   \ 0 \\
   	 ∀ S ∈ \B_k\colon  \qquad \alpha_{S}  &\in    \ \mathbb{R} \\
	 ∀ S ∈ \mathcal{L}_k\colon \qquad  \beta_{S},\beta_{S}^x  &\geq   \ 0
\end{eqnarray*}


\section{Nontrivial steps: Bounding the dual}

\lem[Matrix lemma]{
Given an $h\times n$ matrix $M$ with columns $z_1,z_2,\dots,z_n$. If $\|z_i\|_2\le 1$ for all $i\in[n]$, then there exists a subspace $W$ of $\mathbb{R}^n$ satisfying:
\begin{inparaenum}
\item $dim(W)\ge \frac{n}{2} $, and
\item $\forall y\in W$, $\|My\|_2^2 \le 2\|y\|_2^2$.
\end{inparaenum}
}

\thm[Applying the matrix lemma]{
Let $\mathcal{V}$ be any finite collection of vectors  $v_1,\ldots,v_{h}$ in $\mathbb{R}^n$, and 
for each $v \in \mathcal{V}$, there is some non-negative multiplier $\beta_v \geq 0$.
Consider the operator  \[ B = \sum_{v \in \mathcal{V}}  \beta_v \left(  vv^T   - 2 \sum_{i=1}^n \langle v,e_i \rangle^2 e_ie_i^T \right)\]
where $e_i$ are the standard basis of $\mathbb{R}^n$.
Then there exists a subspace $W$ of dimension at least $n/2$ such that  $\langle y,By \rangle \leq 0$ for every $y \in W$, or equivalently $y^TBy \leq 0$  for every $y \in W$.
}

\lem[$B$ to $B_k$]{
Let

\[ B_k=\sum_{S \in \mathcal{L}_k} \left(\beta_{S} (v_{S}v_{S}^T-2I_{S})+\beta_{S}^x(x_{S}x_{S}^T-2I_{S})\right).\]

If there is a subspace $W$ of dimension at least $n/2$ where $B \preceq 0$, then on the same subspace for all $y\in W$, $y^TB_ky\le 0$.
}

\thm{At time step $k$, the dual program has value $≥ |A_k|/3$.}

\prf[Sketch]{Let $W_1 ≡$ subspace orthogonal to $C = \text{span}\{v_S \mid S ∈ \B_k\}$.

$W_0 ≡ $ the subspace promised by the good subspace lemma. $W ≡ W_1 ∩
W_0$. Consider $P_W$ projection to $W$. The $\alpha$ term disappears
under this projection. Finally, we have:

\begin{align*}
\sum_{i\in A_k}b_i  & = \textrm{Tr}\left[\sum_{i\in A_k} b_ie_ie_i^T \right] \ge \textrm{Tr}\left[P_W\left(\sum_{i\in A_k} b_ie_ie_i^T \right)\right] \\
&  \ge  \textrm{Tr}[P_W]=\textrm{dim}(W) \ge |A_k|/3  \\
\end{align*}

}

\section{Local vs. global energy using martingales}
\dfn[Signed discrepancy]{
Let $D_S(k)$ denote the signed discrepancy of set $S\in\mathcal{S}$ at end of time step $k$ i.e.~$D_S(k)\equiv\sum_{i\in S}x_k(i)$.
}

\dfn{We say that 
$S$ becomes {\em active} at time $k$ if $k$ is the first time step when $|S\cap A_k|\le at$.
}

\thm[Main discrepancy bound]{
Fix a set $S\in\mathcal{S}$.
% and $\lambda\ge 0$ satisfying $\lambda\gamma n^{3/2}<1$. 
Then, for any $\lambda \geq 0 $, the discrepancy of $S$ at time step $T$ satisfies 
\[ \Pr \left[ |D_S(T)| \geq \lambda\sqrt{t} \right] \leq 8 \exp(-\lambda^2/(100a)).\]
}

\dfn[Martingale]{
Let $X_1,X_2,\ldots,X_n$ be a sequence of independent random variables on some probability space, and let 
$Y_k$ be a function of $X_1,\ldots,X_k$. The sequence 
$Y_0,Y_1,Y_2,\ldots,Y_n$ is called a martingale with respect to the sequence $X_1,\ldots,X_n$ if for all $k \in [n]$, 
$\E[|Y_k|]$ is finite and $\E[Y_k| X_1,X_2,...,X_{k-1}]= Y_{k-1}$. We will use $\E_{k-1}[Z]$ to denote $\E[Z | X_1,X_2,...,X_{k-1}]$ where $Z$ is any random variable.
}

\thm[Freedman's inequality]{
Let $Y_0,\ldots,Y_n$ be a martingale with respect to $X_1,\ldots,X_n$ such that 
$|Y_k - Y_{k-1}| \leq M$ for all $k$, and let 
\begin{align*}
W_k &= \sum_{j=1}^k \E_{j-1}[(Y_j - Y_{j-1})^2] = \sum_{j=1}^k\textrm{Var}[Y_j|X_{1},\ldots,X_{j-1}].
\end{align*}
Then for all $\lambda \geq 0$  and $\sigma^2 \geq 0$, we have 
\begin{align*}
 \Pr[|Y_n - Y_0| \geq \lambda \textrm{ and } W_n \leq \sigma^2] 
 \le 2 \exp\left(-\frac{\lambda^2}{2 (\sigma^2 + M \lambda/3)} \right).  
\end{align*}
}

\obs[$D_S(k)$ is a martingale]{
After $S$ becomes active, $D_S(k)$ behaves like a martingale with variance of increment at time step $k$  bounded by
\[ \E_{k-1}[ (D_S(k)-D_S(k-1))^2]  \leq  2 \gamma^2 \sum_{i\in S\cap A_k} \|u_i^k \|_2^2 . \]
}

\dfn[Energy]{ Let \emph{energy} of $S$ at time $k$ be $E_S(k) ≡ ∑_{i ∈ S} x_k(i)^2$.
}

\obs[Energy contribution]{The energy change at time $k$ is a random variable given by
\[ \Delta_kE_S = \gamma^2\sum_{i\in S} \langle r_k,u_i^k\rangle^2+2\gamma\langle r_k,\sum_{i\in S}x_{k-1}(i)u_i^k\rangle \]
}
The first term will be the \emph{quadratic energy change} and the
second will be the \emph{linear energy change}, i.e., 

\[\Delta_kQ_S ≡ \gamma^2\sum_{i\in S} \langle r_k,u_i^k\rangle^2\]
\[ \Delta_kL_S ≡ 2\gamma\langle r_k,\sum_{i\in S}x_{k-1}(i)u_i^k\rangle\]

\end{multicols}
\end{document}
